% file: ML.tex
% Machine Learning (notes), in unconventional ``grande'' format; fitting a widescreen format
% 
% github        : ernestyalumni
% gmail         : ernestyalumni 
% linkedin      : ernestyalumni 
% wordpress.com : ernestyalumni
%
% This code is open-source, governed by the Creative Common license.  Use of this code is governed by the Caltech Honor Code: ``No member of the Caltech community shall take unfair advantage of any other member of the Caltech community.'' 

\documentclass[10pt]{amsart}
\pdfoutput=1
\usepackage{mathtools,amssymb,lipsum,caption}

\usepackage{graphicx}
\usepackage{hyperref}
\usepackage[utf8]{inputenc}
\usepackage{listings}
\usepackage[table]{xcolor}
\usepackage{pdfpages}
\usepackage{tikz}
\usetikzlibrary{matrix,arrows}

\usepackage{multicol}

\hypersetup{colorlinks=true,citecolor=[rgb]{0,0.4,0}}

\oddsidemargin=15pt
\evensidemargin=5pt
\hoffset-45pt
\voffset-55pt
\topmargin=-4pt
\headsep=5pt
\textwidth=1120pt
\textheight=595pt
\paperwidth=1200pt
\paperheight=700pt
\footskip=40pt








\newtheorem{theorem}{Theorem}
\newtheorem{corollary}{Corollary}
%\newtheorem*{main}{Main Theorem}
\newtheorem{lemma}{Lemma}
\newtheorem{proposition}{Proposition}

\newtheorem{definition}{Definition}
\newtheorem{remark}{Remark}

\newenvironment{claim}[1]{\par\noindent\underline{Claim:}\space#1}{}
\newenvironment{claimproof}[1]{\par\noindent\underline{Proof:}\space#1}{\hfill $\blacksquare$}

%This defines a new command \questionhead which takes one argument and
%prints out Question #. with some space.
\newcommand{\questionhead}[1]
  {\bigskip\bigskip
   \noindent{\small\bf Question #1.}
   \bigskip}

\newcommand{\problemhead}[1]
  {
   \noindent{\small\bf Problem #1.}
   }

\newcommand{\exercisehead}[1]
  { \smallskip
   \noindent{\small\bf Exercise #1.}
  }

\newcommand{\solutionhead}[1]
  {
   \noindent{\small\bf Solution #1.}
   }


\title{Machine Learning}
\author{Ernest Yeung \href{mailto:ernestyalumni@gmail.com}{ernestyalumni@gmail.com}}
\date{24 avril 2016}
\keywords{Machine Learning, statistical inference, statistical inference learning}
\begin{document}

\definecolor{darkgreen}{rgb}{0,0.4,0}
\lstset{language=Python,
 frame=bottomline,
 basicstyle=\scriptsize,
 identifierstyle=\color{blue},
 keywordstyle=\bfseries,
 commentstyle=\color{darkgreen},
 stringstyle=\color{red},
 }
%\lstlistoflistings

\maketitle

\tableofcontents


\begin{multicols*}{2}

\begin{abstract}
Everything about Machine Learning.  
\end{abstract}

\part{Introduction}

\subsubsection{Terminology} \quad \\ 
inputs $\equiv $ independent variables $\equiv $ predictors (cf. statistics) $ \equiv $ features (cf. pattern recognition) \\
outputs $\equiv $ dependent variables $\equiv $ responses

cf. Chapter 2 Overview of Supervised Learning, Section 2.1 Introduction of Hastie, Tibshirani, and Friedman (2009) \cite{HTF2009}

cf. Chapter 2 Overview of Supervised Learning, Section 2.2 Variable Types and Terminology  of Hastie, Tibshirani, and Friedman (2009) \cite{HTF2009}

\subsubsection{$\text{FinSet}$} \quad \\ 
The category $\text{FinSet} \in \mathbf{\text{Cat}}$ is the category of all finite sets (i.e. $\text{Obj}(\text{FinSet}) \equiv $ all finite sets) and all functions in between them; note that $\text{FinSet} \subset \mathbf{\text{Set}}$ \footnote{nlab $\text{FinSet}$ \url{https://ncatlab.org/nlab/show/FinSet}}

Recall that the $\text{FinSet}$ \emph{skeletal} is




\end{multicols*}
\begin{thebibliography}{9}
\bibitem{HTF2009}
Trevor Hastie, Robert Tibshirani, Jerome Friedman.   \textbf{The Elements of Statistical Learning: Data Mining, Inference, and Prediction}, Second Edition (Springer Series in Statistics) 2nd ed. 2009. Corr. 7th printing 2013 Edition.  ISBN-13: 978-0387848570.  \url{https://web.stanford.edu/~hastie/local.ftp/Springer/OLD/ESLII_print4.pdf}



\end{thebibliography}

\end{document}
